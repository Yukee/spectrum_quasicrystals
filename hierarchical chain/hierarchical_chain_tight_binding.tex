\documentclass[11pt]{article}

\usepackage[utf8]{inputenc}
\usepackage[T1]{fontenc}
\usepackage[english, french]{babel} %français
\usepackage{amsmath}
\usepackage{amsfonts}
\usepackage{makeidx}
\usepackage{graphicx}
\usepackage[left=2cm,right=2cm,top=2cm,bottom=2cm]{geometry}
\usepackage{mathtools} %dcases
\usepackage{braket} %quantum mechanics
\usepackage[colorlinks=true, linkcolor=black, citecolor=black]{hyperref} % hyperlinks
\usepackage{tikz} % drawing in LaTeX

% the equal sign I use to define something
\newcommand{\define}{\ensuremath{ \overset{\text{def}}{=} }}

% differential element
\renewcommand{\d}[1]{\mathrm{d}#1}

% Hermitian conjugate
\DeclareMathOperator{\hc}{h.c.}

\title{\textbf{The hierarchical chain (tight binding model)}}
\author{}
\date{October 9, 2014}
\begin{document}

% I'm writing in english
\selectlanguage{english}

\maketitle

\section{The model}

\begin{center}
    	\begin{tikzpicture}[scale=1]
    		\newcommand{\orig}{-1.5}
    		\newcommand{\trans}{2}
    	
    		% bonds 
        	\draw[-] (\orig, 0) -- (\orig+\trans, 0) node [midway, above] {$t_0$};
			\draw[-,dashed] (\orig+\trans,0) -- (\orig+2*\trans,0) node [midway, above] {$t_1$};
			\draw[-] (\orig+2*\trans,0) -- (\orig+3*\trans,0) node [midway, above] {$t_0$};			
    	
    		% sites
			\filldraw (\orig,0) circle (0.05) node [below] {$\ket{0}$};
			\filldraw (\orig+\trans,0) circle (0.05) node [below] {$\ket{1}$};
			\filldraw (\orig+2*\trans,0) circle (0.05) node [below] {$\ket{2}$};
			\filldraw (\orig+3*\trans,0) circle (0.05) node [below] {$\ket{3}$};

		\end{tikzpicture}
\end{center}

We consider the tight-binding Hamiltonian:
\begin{equation}
	H = \sum_i t(i) \ket{i-1} \bra{i} + \hc
\end{equation}
where
\begin{equation}
	t(i) \define t_k\text{, with $k$ the number of times 2 divides $i$.}
\end{equation}
Such a Hamiltonian has a the structure of a binary tree. This binary hierarchy must somehow show up in its spectrum, at least given a reasonable set of $t_k$s.

\subsection{$t_0 \gg t_1 \gg ... \gg t_1^2 \gg t_2^2 \gg ... \gg t_1^3 \gg t_2^3 \gg ...$}
In that case, we can treat the problem perturbatively ; first integrate out the diatomic molecules (two atoms liked by a $t_0$ bond), then the the diatomic molecules of diatomic molecules (two diatomic molecules linked by a $t_1$ bond), etc.

At the first order in perturbation theory the spectrum is
\begin{equation}
	E_{\epsilon_1\epsilon_2...} = (-1)^{\epsilon_1} t_0 + (-1)^{\epsilon_2} \frac{t_1}{2} + (-1)^{\epsilon_3} \frac{t_2}{4} + ...
\end{equation} 
At zeroth order in
\end{document}
