\documentclass[11pt]{article}

\usepackage[utf8]{inputenc}
\usepackage[T1]{fontenc}
\usepackage[english, french]{babel} %français
\usepackage{amsmath}
\usepackage{amsfonts}
\usepackage{makeidx}
\usepackage{graphicx}
\usepackage[left=2cm,right=2cm,top=2cm,bottom=2cm]{geometry}
\usepackage{mathtools} %dcases
\usepackage{braket} %quantum mechanics
\usepackage[colorlinks=true, linkcolor=black, citecolor=black]{hyperref} % hyperlinks
\usepackage{tikz} % drawing in LaTeX
\usepackage{ dsfont } % hollow letters
% the equal sign I use to define something
\newcommand{\define}{\ensuremath{ \overset{\text{def}}{=} }}

% differential element
\renewcommand{\d}[1]{\mathrm{d}#1}

% similar symbol with a limit underneath
\newcommand{\simlim}[2]{\ensuremath{ \underset{#1 \rightarrow #2}{\sim} }}

\newcommand{\om}{\ensuremath{\omega}}
\newcommand{\lb}{\ensuremath{\lambda}}
\newcommand{\lbbar}{\ensuremath{\overline{\lambda}}}
\newcommand{\zbar}{\ensuremath{\overline{z}}}

\title{\textbf{Fractal dimensions of quasicrystals \emph{via} a perturbative renormalization group.}}
\author{}
\date{}
\begin{document}

\selectlanguage{english}

\maketitle

\abstract{}

In this paper, we focus on tight-binding Hamiltonians on one-dimensional quasiperiodic tilings.
Notable examples include the Harper model, and the family of quasiperiodic Hamiltonians constructed by the cut and project method. 
Each of the latter is associated with an irrational number, $\alpha$.
It has the geometrical interpretation of the tangent of the angle between the projection axis and the direction of one of the basis vectors of the two-dimensional superlattice.
In such a model, if $\psi_i$ is the coefficient at site $i$ of the wavefunction associated to the energy $E$, we have
\begin{equation}
	E \psi_i = t^\alpha_i \psi_{i-1} + t^\alpha_{i+1} \psi_{i+1}
\end{equation}
where the jump amplitudes $t^\alpha_i$ can take two values $t_s$, $t_w$:
\begin{equation}
	t_i^\alpha = \begin{dcases*}
	t_w & when $i \bmod(1+\alpha) \geq \alpha$ \\
	t_s & otherwise.
	\end{dcases*}
\end{equation}
If we replace the irrational $\alpha$ by a rational approximation $\alpha_n = p_n/q_n$, we obtain a periodic system of period $p_n + q_n$. 
Thus, to a sequence of rationals $\{\alpha_n\}_n$ converging to $\alpha$ is associated in a natural way a sequence of periodic tight-binding Hamiltonians -- called approximants, converging to a quasiperidic Hamiltonian.
It is convenient to consider such a sequence of periodic systems. In particular we will be able to find reccurence relations for the fractal dimensions in that way.


Amongst all irrationals, the golden ratio and its inverse, $\omega = 2/(1+\sqrt{5})$, play a special role. They contain only the number 1 in their continued fraction expansion. In that sense, these two numbers are the hardest to approximate by rationnals. 
We thus expect the quasiperiodicity to have the most spectacular consequences when the irrational is taken to be the golden ratio or its inverse.

We restrict ourselves to the case $\alpha \leq 1$ (the other case being equivalent up the exchange of $t_s$ and $t_w$). As an example, we are going to chose $\alpha = \omega$. The tight-binding Hamiltonian resulting from this choice is called the Fibonacci Hamiltonian.
However, our results are general and apply to every Hamiltonian constructed by the cut and project method.

\section{Co-numbering and energy levels.}




\bibliography{fractal_dimensions_quasicrystals.bib}{}
\bibliographystyle{plain}
\end{document}