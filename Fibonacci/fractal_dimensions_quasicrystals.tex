\documentclass[11pt]{article}

\usepackage[utf8]{inputenc}
\usepackage[T1]{fontenc}
\usepackage[english, french]{babel} %français
\usepackage{amsmath}
\usepackage{amsfonts}
\usepackage{makeidx}
\usepackage{graphicx}
\usepackage[left=2cm,right=2cm,top=2cm,bottom=2cm]{geometry}
\usepackage{mathtools} %dcases
\usepackage{braket} %quantum mechanics
\usepackage[colorlinks=true, linkcolor=black, citecolor=black]{hyperref} % hyperlinks
\usepackage{tikz} % drawing in LaTeX
\usepackage{ dsfont } % hollow letters
% the equal sign I use to define something
\newcommand{\define}{\ensuremath{ \overset{\text{def}}{=} }}

% differential element
\renewcommand{\d}[1]{\mathrm{d}#1}

% similar symbol with a limit underneath
\newcommand{\simlim}[2]{\ensuremath{ \underset{#1 \rightarrow #2}{\sim} }}

\newcommand{\om}{\ensuremath{\omega}}
\newcommand{\lb}{\ensuremath{\lambda}}
\newcommand{\lbbar}{\ensuremath{\overline{\lambda}}}
\newcommand{\zb}{\ensuremath{\overline{z}}}

\title{\textbf{Fractal dimensions of quasicrystals \emph{via} a perturbative renormalization group.}}
\author{}
\date{}
\begin{document}

\selectlanguage{english}

\maketitle

\abstract{}

In this paper, we focus on tight-binding Hamiltonians on one-dimensional quasiperiodic tilings.
Notable examples include the Harper model, and the family of quasiperiodic Hamiltonians constructed by the cut and project method. 
Each of the latter is associated with an irrational number, $\alpha$.
It has the geometrical interpretation of the tangent of the angle between the projection axis and the direction of one of the basis vectors of the two-dimensional superlattice.
The Hamiltonian of such a model writes
\begin{equation}
	H^\alpha = \sum_i t^\alpha_i \left( \ket{i} \bra{i+1} + \ket{i+1} \bra{i} \right)
\end{equation}
where the jump amplitudes $t^\alpha_i$ can take two values $t_s$, $t_w$:
\begin{equation}
	t_i^\alpha = \begin{dcases*}
	t_w & when $i \bmod(1+\alpha) \geq \alpha$, \\
	t_s & otherwise.
	\end{dcases*}
\end{equation}
If we replace the irrational $\alpha$ by a rational approximation $\alpha_n = p_n/q_n$, the sequence of couplings is modified:
\begin{equation}
	t_i^{p_n/q_n} = \begin{dcases*}
	t_w & when $q_n i \bmod(p_n+q_n) \geq p_n$, \\
	t_s & otherwise,
	\end{dcases*}
\end{equation}
and we obtain a periodic system of period $p_n + q_n$. 
Thus, to a sequence of rationals $\{\alpha_n\}_n$ converging to $\alpha$ is associated in a natural way a sequence of periodic tight-binding Hamiltonians -- called approximants, converging to a quasiperidic Hamiltonian. We will call $H_n$ the $n^\text{th}$ approximant, generated by the rationnal $\alpha_n$.


Amongst all irrationals, the golden ratio and its inverse, $\omega = 2/(1+\sqrt{5})$, play a special role. They contain only the number 1 in their continued fraction expansion. In that sense, these two numbers are the hardest to approximate by rationnals. 
We thus expect the quasiperiodicity to have the most spectacular consequences when the irrational is taken to be the golden ratio or its inverse.

We restrict ourselves to the case $\alpha \leq 1$ (the other case being equivalent up the exchange of $t_s$ and $t_w$). As an example, we are going to chose $\alpha = \omega$. The tight-binding Hamiltonian resulting from this choice is called the Fibonacci Hamiltonian.
However, our results are general and apply to every Hamiltonian constructed by the cut and project method.

\section{Co-numbering, atoms and molecules.}

We consider a periodic approximant given by the rational $\alpha_n = p_n/q_n$. Later, we are going to specialize to the case of the Fibonacci Hamiltonian, but for the moment we wish to stay as general as possible.
We have already seen already that the integer
\begin{equation}
	i'_k = q_n k \bmod(p_n+q_n)
\end{equation}
determines the sequence of jump amplitudes. How does $i'_k$ changes when we jump from one site to the next, i.e. when we increase $k$ by one unit? It is easy to check that
\begin{equation}
	i'_{k+1} = \begin{dcases*}
	i'_k + p_n & when sites $k$ and $k+1$ are linked by $t_w$, \\
	i'_k + q_n & when sites $k$ and $k+1$ are linked by $t_s$.
	\end{dcases*}
\end{equation}
Thus, the sequence of $i'_k$ furnishes a natural renumbering of the sites.
In the basis where sites are numbered using $i'$, up to a suitable shift of the origin, the Hamiltonian rewrites as a long-range, Toeplitz matrix:
\begin{equation}
	H_n = 
	\bordermatrix{ 
	 	& 1 	&	\ldots & & p_n	& &  \ldots &	& q_n &	& \ldots	&  \cr
    1 	& 0 		& \ldots & 0 & t_w & 0	& \ldots & 0 & t_s	& 0 		& \ldots		 \cr
    \vdots & & & & & \ddots	& & & & \ddots & \cr
    p_n & t_w \cr
    \vdots & & \ddots \cr
    q_n & t_s \cr
    \vdots & & \ddots \cr
     & 
    }
\end{equation}
The co-numbering allows to distinguish two classes of sites. We call \emph{molecular} the first $p_n$ and the last $p_n$ sites. We call \emph{atomic} the remaining $q_n - p_n$ sites.
Each molecular site is coupled to another molecular site by a $t_s$ coupling, and to an atomic site by a $t_w$ coupling. Thus, in the limit $t_w \ll t_s$, molecular sites form isolated diamtomic \emph{molecules}. 
On the other hand, the atomic sites are all coupled to two molecular sites by $t_w$ couplings. Thus, in the limit $t_w \ll t_s$, they form isolated \emph{atoms}.

Now, we focus on the particular case of the Fibonacci Hamiltonian. 
We take $p_n = F_{n-2}$, $q_n = F_{n-1}$ where $F_n$ is the $n^\text{th}$ Fibonacci number. Then, $\alpha_n = F_{n-2}/F_{n-1}$ indeed approximates the inverse golden ratio, so that we have constructed an approximant to the Fibonacci chain.
The $n^\text{th}$ Fibonacci approximant consists of a block of $F_{n-2} + F_{n-1} = F_{n}$ sites, repeated periodically. That block contain $F_{n-1} - F_{n-2} = F_{n-3}$ atoms and $F_{n-2}$ molecules (that is, $2F_{n-2}$ molecular sites).

Figure \eqref{fig:fib8} shows the molecules and atoms of the fifth Fibonacci approximant, together with the co-numbering of the sites.

\begin{figure}[htp]
\centering
    	\begin{tikzpicture}[scale=.7]
    		\newcommand{\orig}{-1.5}
    		\newcommand{\trans}{1.5}
    		\newcommand{\vertspac}{-2.}
    	
    		% initial chain
    	
    		% bonds 
        	\draw[-,double] (\orig, 0)  node [left] {$F_{5} = 8$}  -- (\orig+\trans, 0);
			\draw[-] (\orig+\trans,0) -- (\orig+2*\trans,0); % node [midway, above] {$t_s$};
			\draw[-,double] (\orig+2*\trans,0) -- (\orig+3*\trans,0); % node [midway, above] {$t_w$};	
			\draw[-] (\orig+3*\trans,0) -- (\orig+4*\trans,0); % node [midway, above] {$t_s$};
			\draw[-] (\orig+4*\trans,0) -- (\orig+5*\trans,0); % node [midway, above] {$t_w$};
			\draw[-,double] (\orig+5*\trans,0) -- (\orig+6*\trans,0); % node [midway, above] {$t_w$};
			\draw[-] (\orig+6*\trans,0) -- (\orig+7*\trans,0); % node [midway, above] {$t_s$};
			\draw[-,double] (\orig+7*\trans,0) -- (\orig+8*\trans,0); % node [midway, above] {$t_w$};
    	
    	
    		% sites
		    \filldraw (\orig+0*\trans,0) circle (0.05) node [below] {1};
		    \filldraw (\orig+1*\trans,0) circle (0.05) node [below] {6};
		    \filldraw (\orig+2*\trans,0) circle (0.05) node [below] {3};
		    \filldraw (\orig+3*\trans,0) circle (0.05) node [below] {8};
		    \filldraw (\orig+4*\trans,0) circle (0.05) node [below] {5};
		    \filldraw (\orig+5*\trans,0) circle (0.05) node [below] {2};
		    \filldraw (\orig+6*\trans,0) circle (0.05) node [below] {7};
		    \filldraw (\orig+7*\trans,0) circle (0.05) node [below] {4};
		      
		\end{tikzpicture}
\caption{The periodically repeated block of the fifth approximant to the Fibonacci chain. Weak couplings $t_w$ are represented by a single line, and strong couplings $t_s$ by a double line. Below each site is its co-numbered label.}
\label{fig:fib8}
\end{figure}

\section{Deflation and renormalization}

\subsection{Deflation, molecular and atomic chains.}
Besides the cut and project method, we can construct the Fibonacci chain by inflation.
We start from the trivial chain:
\begin{equation}
	C_0 = t_s
\end{equation}
and we apply repetively the \emph{inflation rule}
\begin{equation}
	r \define \begin{cases}
        t_{w} & \rightarrow t_w t_s \\
        t_s & \rightarrow t_w
      \end{cases}
\end{equation} 
on it to build new chains: $C_1 = r(C_0) = t_w$, $C_2 = r(C_1) = t_w t_s$, ... $C_n = r^n(C_0)$.
Then, perhaps up to a global circular permutation of the couplings, the infinite chain $C_n C_n C_n \dots$ is the sequence of couplings of the $n^\text{th}$ approximant.
Furthermore, we can define \emph{deflation rules} relating an approximant to a smaller one.
The \emph{molecular deflation rule}
\begin{equation}
	d_m = \begin{cases}
        t_{w} & \leftarrow t_s t_w t_w (t_s)\\
        t_s & \leftarrow t_s t_w (t_s)
      \end{cases}
\end{equation}
decimates all sites except molecular ones. Fig. \eqref{fig:mol_defl} examplifies the decimation operation.
Crucially, the deflated chain is again a Fibonacci chain. More precisely, the molecular deflation relates the approximant of size $n$ to the approximant of size $n-2$: $d_m(C_n C_n \dots) = C_{n-2} C_{n-2} \dots$.

\begin{figure}[htp]
\centering
    	\begin{tikzpicture}[scale=.7]
    		\newcommand{\orig}{-1.5}
    		\newcommand{\trans}{1.5}
    		\newcommand{\vertspac}{-2.}
    		\newcommand{\vertsize}{.5} % vertical spand of the rectangles
    		\newcommand{\del}{.2}
    	
    		% initial chain
    	
    		% bonds 
        	\draw[-] (\orig, 0)  node [left] {}  -- (\orig+\trans, 0);
			\draw[-,double] (\orig+\trans,0) -- (\orig+2*\trans,0); % node [midway, above] {$t_s$};
			\draw[-] (\orig+2*\trans,0) -- (\orig+3*\trans,0); % node [midway, above] {$t_w$};	
			\draw[-,double] (\orig+3*\trans,0) -- (\orig+4*\trans,0); % node [midway, above] {$t_s$};
			\draw[-] (\orig+4*\trans,0) -- (\orig+5*\trans,0); % node [midway, above] {$t_w$};
			\draw[-] (\orig+5*\trans,0) -- (\orig+6*\trans,0); % node [midway, above] {$t_w$};
			\draw[-,double] (\orig+6*\trans,0) -- (\orig+7*\trans,0); % node [midway, above] {$t_s$};
			\draw[-] (\orig+7*\trans,0) -- (\orig+8*\trans,0); % node [midway, above] {$t_w$};
    	
    		% sites
			\foreach \x in {0,...,7}
		      \filldraw (\orig+\x*\trans,0) circle (0.05); % node [below] {$\ket{\x}$};
		      
		    % rectangles around molecules
		    \draw [rounded corners] (\orig +\trans-\del,-\vertsize) rectangle (\orig+2*\trans+\del,\vertsize);
		    \draw [rounded corners] (\orig +3*\trans-\del,-\vertsize) rectangle (\orig+4*\trans+\del,\vertsize);
		    \draw [rounded corners] (\orig +6*\trans-\del,-\vertsize) rectangle (\orig+7*\trans+\del,\vertsize);
		    
		    % arrows below rectangles
		    \draw [->] (\orig+1.5*\trans,-\vertsize-\del) -- (\orig+1.5*\trans,\vertspac+\del);
		    \draw [->] (\orig+3.5*\trans,-\vertsize-\del) -- (\orig+3.5*\trans,\vertspac+\del);
		    \draw [->] (\orig+6.5*\trans,-\vertsize-\del) -- (\orig+6.5*\trans,\vertspac+\del);
		      
			% molecular chains
			
			\foreach \x in {1}
			{
				\draw[-] (\orig, \x*\vertspac) node [left] {} -- (\orig+1.5*\trans, \x*\vertspac);
				\draw[-,double] (\orig+1.5*\trans, \x*\vertspac) -- (\orig+3.5*\trans, \x*\vertspac);
				\draw[-] (\orig+3.5*\trans, \x*\vertspac) -- (\orig+6.5*\trans, \x*\vertspac);
				\draw[-,double] (\orig+6.5*\trans, \x*\vertspac) -- (\orig+8*\trans, \x*\vertspac);
				
				\filldraw (\orig+1.5*\trans,\x*\vertspac) circle (0.05);
				\filldraw (\orig+3.5*\trans,\x*\vertspac) circle (0.05);
				\filldraw (\orig+6.5*\trans,\x*\vertspac) circle (0.05);
			}
		\end{tikzpicture}
\caption{The molecular deflation rule illustrated. Here we relate the fifth approximant to the third.}
\label{fig:mol_defl}
\end{figure}

Similarly, we can perform a decimation operation on all sites except atomic ones. We call such an operation an \emph{atomic decimation}.
Again, the deflated chain is also a Fibonacci chain.
The atomic decimation relates the approximant of size $n$ to the approximant of size $n-3$. 
Fig. \eqref{fig:at_defl} examplifies the procedure.

\begin{figure}[htp]
\centering
    	\begin{tikzpicture}[scale=.7]
    		\newcommand{\orig}{-1.5}
    		\newcommand{\trans}{1.5}
    		\newcommand{\vertspac}{-2.}
    		\newcommand{\vertsize}{.5} % vertical spand of the rectangles
    		\newcommand{\del}{.2}
    	
    		% initial chain
    	
    		% bonds 
        	\draw[-] (\orig, 0)  node [left] {}  -- (\orig+\trans, 0);
			\draw[-,double] (\orig+\trans,0) -- (\orig+2*\trans,0); % node [midway, above] {$t_s$};
			\draw[-] (\orig+2*\trans,0) -- (\orig+3*\trans,0); % node [midway, above] {$t_w$};	
			\draw[-,double] (\orig+3*\trans,0) -- (\orig+4*\trans,0); % node [midway, above] {$t_s$};
			\draw[-] (\orig+4*\trans,0) -- (\orig+5*\trans,0); % node [midway, above] {$t_w$};
			\draw[-] (\orig+5*\trans,0) -- (\orig+6*\trans,0); % node [midway, above] {$t_w$};
			\draw[-,double] (\orig+6*\trans,0) -- (\orig+7*\trans,0); % node [midway, above] {$t_s$};
			\draw[-] (\orig+7*\trans,0) -- (\orig+8*\trans,0); % node [midway, above] {$t_w$};
    	
    		% sites
			\foreach \x in {0,...,7}
		      \filldraw (\orig+\x*\trans,0) circle (0.05); % node [below] {$\ket{\x}$};
		      
		    % rectangles around atoms
		    \draw [rounded corners] (\orig-\vertsize,-\vertsize) rectangle (\orig+\vertsize,\vertsize);
		    \draw [rounded corners] (\orig-\vertsize+5*\trans,-\vertsize) rectangle (\orig+\vertsize+5*\trans,\vertsize);
		    
		    % arrows below rectangles
		    \draw [->] (\orig,-\vertsize-\del) -- (\orig,\vertspac+\del);
		    \draw [->] (\orig+5*\trans,-\vertsize-\del) -- (\orig+5*\trans,\vertspac+\del);
		      
		    % atomic chain
		    
        	\draw[-] (\orig, \vertspac)  node [left] {}  -- (\orig+5*\trans, \vertspac);
			\draw[-,double] (\orig+5*\trans,\vertspac) -- (\orig+8*\trans,\vertspac); % node [midway, above] {$t_s$};
			
			\filldraw (\orig,\vertspac) circle (0.05); % node [below] {$\ket{\x}$};
			\filldraw (\orig+5*\trans,\vertspac) circle (0.05); % node [below] {$\ket{\x}$};
%			\filldraw (\orig+8*\trans,\vertspac) circle (0.05); % node [below] {$\ket{\x}$};
		\end{tikzpicture}
\caption{The atomic deflation rule illustrated. Here we relate the fifth approximant to the second.}
\label{fig:at_defl}
\end{figure}

\subsection{Renormalization}

We are now going to focus on the limit $t_w \ll t_s$, in which the disinction between atomic and molecular sites acquires its real meaning.
We define $\rho = t_w/t_s$. It will be our perturbative parameter.
The energies are at most of order $t_s$. As varying $t_s$ (while keeping $\rho$ fixed) simply amounts to rescaling the whole energy spectrum, we arbitrarily set $t_s = 1$. 

When $\rho = 0$, the atoms and the molecules decouple. The eigenstates are the molecular bonding and antibonding states, at energies $\pm 1$, and the atomic state at zero energy.
The spectrum is constituted of three infinitely degenerated levels.

When $\rho > 0$, $\rho \ll 1$, perturbation theory tells us that states inside each of the three degenerated levels weakly coupled to each other, thus raising the degeneracy.
Let us focus to begin with on atomic states. At first order, each atomic site is coupled to the neighbouring atomic sites.
Effectively, we can therefore work on the \emph{deflated} chain, with effective hopping amplitudes coupling atomic sites. Perturbation theory gives us the expression of the effective hopping amplitudes \cite{Niu1990}.
We find
\begin{equation}
	\begin{dcases}
	t^\text{atomic}_w &= \zb \rho\\
	t^\text{atomic}_s &= \zb,
	\end{dcases}
\end{equation}
with $\zb =\rho^2$.

Similarly, each molecular site is coupled to the neighbouring molecular sites. 
There is here a small subtlety. A molecule sits on two neighbouring sites. Say for example that that sites $i$ and $i+1$ on the chain form a molecule.
Then, by a change of basis, we can say that the molecule sits at the ``bonding'' and ``antibonding'' sites whose position is respectively given by the linear combination of the localized states $\ket{i} + \ket{i+1}$ and $\ket{i} - \ket{i+1}$.
It is natural to do that, because when $\rho = 0$ the molecular eigenstates are localized in this new basis.
At first order, the bonding sites couple to each other. 
This results in effective hopping amplitudes given by
\begin{equation}
	\begin{dcases}
	t^\text{bonding}_w &= z \rho\\
	t^\text{bonding}_s &= z,
	\end{dcases}
\end{equation}
with $z = \rho/2$. This also results in the appearance of an onsite potential, $V^\text{bonding}  = -1$. 
The antibonding sites similarly couple to each other, resulting in the same effective hopping terms, but in a different onsite potential, $V^\text{antibonding} = +1$.

To summarize, we have seen that we can formally separate the chain of the $n^\text{th}$ approximant into molecular and atomic chains.
In the limit $\rho \ll 1$, the Hamiltonian of the $n^\text{th}$ approximant decouples into the direct sum of three Hamiltonian: an atomic Hamiltonian living on the atomic chain, a bonding Hamiltonian living on the molecular chain, and an antibonding Hamiltonian also living on the molecular chain. 
Because the atomic and molecular chains are again Fibonacci chains (but of smaller lengths), we can relate these three Hamiltonians to Fibonacci Hamiltonians, with renormalized hopping terms and onsite energies.
We can summarize these results in an equation:
\begin{equation}
	H_n = \left( z H_{n-2} - 1 \right) \oplus \left( \zb H_{n-3} \right) \oplus \left( z H_{n-2} + 1 \right) 
\end{equation}

In the limit $n \rightarrow \infty$, the chain becomes quasiperiodic. As such, we expect its wavefunctions and its spectrum to be nontrivial, namely to exhibit multifractality.
We are going to try using the recursion relation we have on the Hamiltonians of the approximants to derive recursion relations on energies and wavefunctions. In this way, we hope to gain some insight on the form of the spectrum and of the wavefunctions in the limit $n \rightarrow \infty$.
From that, we hope to characterize the multifractality of the quasiperiodic chain by computing the fractal dimensions of its wavefunctions and of its spectrum.

\subsection{Renormalization paths, equivalence between energy labels and co-numbers.}

The co-numbering naturally distinguishes between molecules and atoms: the first $p_n$ sites are the left part of a molecule, the next $q_n - p_n$ sites are the atomic sites, and the last $p_n$ sites are the right part of a molecule.

\bibliography{fractal_dimensions_quasicrystals.bib}{}
\bibliographystyle{plain}
\end{document}