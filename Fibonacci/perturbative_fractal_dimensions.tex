\documentclass[11pt]{article}

\usepackage[utf8]{inputenc}
\usepackage[T1]{fontenc}
\usepackage[english, french]{babel} %français
\usepackage{amsmath}
\usepackage{amsfonts}
\usepackage{makeidx}
\usepackage{graphicx}
\usepackage[left=2cm,right=2cm,top=2cm,bottom=2cm]{geometry}
\usepackage{mathtools} %dcases
\usepackage{braket} %quantum mechanics
\usepackage[colorlinks=true, linkcolor=black, citecolor=black]{hyperref} % hyperlinks
\usepackage{tikz} % drawing in LaTeX
\usepackage{ dsfont } % hollow letters
% the equal sign I use to define something
\newcommand{\define}{\ensuremath{ \overset{\text{def}}{=} }}

% differential element
\renewcommand{\d}[1]{\mathrm{d}#1}

% similar symbol with a limit underneath
\newcommand{\simlim}[2]{\ensuremath{ \underset{#1 \rightarrow #2}{\sim} }}

\newcommand{\om}{\ensuremath{\omega}}
\newcommand{\lb}{\ensuremath{\lambda}}
\newcommand{\lbbar}{\ensuremath{\overline{\lambda}}}
\newcommand{\zbar}{\ensuremath{\overline{z}}}

\title{\textbf{Fractal dimensions of the Fibonacci chain \emph{via} a perturbative renormalization group.}}
\author{Nicolas Macé}
\date{9 décembre 2014}
\begin{document}

\selectlanguage{english}

\maketitle

\section{Notations}

\subsection{General}
\begin{itemize}
	\item Indices $a,b,c...$ will be used to label energy bands, while indices $i,j,k...$ will label sites.
	\item The superscript $n$ is used to denote the $n^\text{th}$ approximant of the Fibonacci chain. This approximant is constituted of $F_n$ sites, with periodic boundary conditions.
\end{itemize}

\subsection{Multifractal analysis}
\begin{itemize}
	\item $\alpha(x)$ is the pointwise Hölder exponent at point $x$, of some fractal set $X$ that could be the spectrum, or the set of components of a wavefunction, etc. $\alpha$ is also the conjugate variable (by a Legendre transform) of $q$. 
	\item $f(\alpha)$ is the Hausdorff dimension of the set $H_\alpha = \{ x \in X | \alpha(x) = \alpha \}$. It is also the Legendre transform of $\tau$.
\end{itemize}

\subsection{Renormalization parameters}
\begin{align}
	z &= \rho/2\\
	\zbar &= \rho^2\\
	\lb &= 1/(2+\rho^2)\\
	\lbbar &= 1/(1+2\rho^2)\\
	c &= 1+\rho^2 \\
	s &= 1-\rho^2
\end{align}
We also introduce the notation $f_q(\rho) = f(\rho^q)$, where $f$ can be any of the above renormalization parameters.

\section{The renormalization equations}

\subsection{Threefold decimation}
We wish to consider the sequence of approximants in the \textit{strong quasiperiodicity limit}: $t_s/t_w \ll 1$.
In view of passing to this perturbative framework, we define the small parameter
\begin{equation}
	\rho \define \frac{t_w}{t_s}
\end{equation}

\textbf{At leading order} in $\rho$ weak bonds disappear: $t_w = 0$. 
Therefore we can diagonalize separately \textit{molecules}, ie atomic sites linked by a strong coupling, and \textit{atoms}, ie atomic sites surrounded by two weak couplings.

\textbf{At next-to-leading order}, two neighbouring molecules/atoms can interact, and the study of the chain $S_n$ is naturally equivalent to the study of an atomic chain $d_a(S_n) = S_{n-3}$ and two molecular chains $d_m(S_n) = S_{n-2}$. 

The \textit{deflation} of the chain $S_n$ translates into a \textit{renormalization} of the Hamiltonian $H_n$: $H_n$ is related to $H_{n-2}$ by the decimation of atomic sites, and to $H_{n-3}$ by the decimation of molecular sites. 
Though the action of the deflation rules $d_a$ and $d_m$ on $S_n$ is simple, the action of the decimation rules on $H_n$ is far from trivial.
In particular, after a decimation, couplings and wavefunctions are renormalized. We are now interested in characterizing this renormalization.

\subsection{Renormalization of the couplings}

\subsubsection{Atomic chain}

\subsubsection{Molecular chains}

\subsection{Renormalization of the wavefunctions}

\subsubsection{Atomic chain}

\begin{center}
    	\begin{tikzpicture}[scale=1]
    		\newcommand{\orig}{-1.5}
    		\newcommand{\trans}{2}
    	
    		% bonds 
        	\draw[-] (\orig, 0) -- (\orig+\trans, 0) node [midway, above] {$t_0$};
			\draw[-,double] (\orig+\trans,0) -- (\orig+2*\trans,0) node [midway, above] {$t_1$};
			\draw[-] (\orig+2*\trans,0) -- (\orig+3*\trans,0) node [midway, above] {$t_0$};			
    	
    		% sites
			\filldraw (\orig,0) circle (0.05) node [below] {$\ket{0}$};
			\filldraw (\orig+\trans,0) circle (0.05) node [below] {$\ket{1}$};
			\filldraw (\orig+2*\trans,0) circle (0.05) node [below] {$\ket{2}$};
			\filldraw (\orig+3*\trans,0) circle (0.05) node [below] {$\ket{3}$};

		\end{tikzpicture}
\end{center}

At leading order in $\rho$, an atomic wavefunction of energy $E$ is sent by $d_a$ to a \emph{not necesarily atomic} wavefunction of energy $E/\bar z$.
If $i$ is the label of an atomic site of the chain $S_n$, we have
\begin{equation}
	\psi_i(E)  = \beta_i\left(\{ \psi_{i'}(E/\bar z) \}_{i'}\right)
\end{equation} 
We assume that the transformation $\beta$ is linear:
\begin{equation}
	\psi_i(E)  = \beta_{ii'}  \psi_{i'}(E/\bar z)
\end{equation}
and even more than that, that it is a simple homothety: $\beta_{ii'} = \sqrt{\lambda} \delta_{d_a(i)i'}$.
We therefore have
\begin{equation}
\label{eq:renorm_a}
	\psi_i(E) = \sqrt{\lambda} \psi_{d_a(i)}(E/\bar z)
\end{equation}
$\lambda$ is constrained by the unitarity of the wavefunction:
\begin{equation}
	\Sigma^n \define \sum_{i=1}^{F_n} |\psi_i|^2 = 1
\end{equation}
Let us note that we can partition the set of atomic sites of $S_n$ into the set of atomic sites that remain atomic sites after deflation $S_{a'}$ and the set atomic sites that become molecular sites after deflation $S_{m'}$.
Using this partition to transform the sum on all sites, we get
\begin{equation}
	\sum_{i=1}^{F_n} |\psi_i|^2 = \sum_{i \in S_{a'}} \left( |\psi_i|^2 + |\psi_{i+2}|^2 + |\psi_{i-2}|^2 + |\psi_{i+4}|^2 + |\psi_{i-4}|^2 \right) + \sum_{i \in S_{m'}} \left( |\psi_i|^2 + |\psi_{i+2}|^2 + |\psi_{i-2}|^2 + |\psi_{i\pm4}|^2 \right)
\end{equation}
We can also partition this sum into a sum over atomic sites $\Sigma^n_a$ and over molecular sites $\Sigma^n_m$.
We have
\begin{equation}
	\begin{cases}
	\Sigma^n_a &= \sum_{i \in S_{a'}}  |\psi_i|^2  + \sum_{i \in S_{m'}}  |\psi_i|^2\\
	\Sigma^n_m &= \sum_{i \in S_{a'}} \left( |\psi_{i+2}|^2 + |\psi_{i-2}|^2 + |\psi_{i+4}|^2 + |\psi_{i-4}|^2 \right) + \sum_{i \in S_{m'}} \left(|\psi_{i+2}|^2 + |\psi_{i-2}|^2 + |\psi_{i\pm4}|^2 \right)
	\end{cases}
\end{equation}
Now, beacause of the way we assumed the wavefunction on atomic sites is renormalized (eq. \eqref{eq:renorm_a}), we have
\begin{align}
	\Sigma_a^n &= \bar \lambda  \left(  \sum_{i \in S_{a'}}  |\psi_{d_a(i)}|^2  + \sum_{i \in S_{m'}}  |\psi_{d_a(i)}|^2 \right) \\
	&= \bar \lambda \sum_{i = 1}^{F_{n-3}} |\psi_i|^2 \\
	&= \bar \lambda \Sigma^{n-3}
\end{align}
and thus
\begin{equation}
\label{eq:renorm_aa}
\boxed{
	\Sigma_a^n = \bar \lambda
	}
\end{equation}

Now, in the perturbative limit, the great thing is that the wafunction at molecular sites can be expressed in terms of the wavefunction at atomic sites\footnote{To derive these relations, we use the fact that $E$ being an atomic energy, it is at most of order $\rho$. We also use the fact that at leading order the wavefunction is of the same order on all atomic sites.}! 
\begin{equation}
	\begin{cases}
	\psi_{i\pm2} &= - \rho \psi_i + \mathcal{O}(\rho E \psi_i) \\
	\psi_{i\pm4} &= \phantom{-}\rho^2 \psi_i  + \mathcal{O}(\rho^4 \psi_i)
	\end{cases}
\end{equation}
Using this together with \eqref{eq:renorm_a} we get\footnote{If energy really is at most of order $\rho$, the formula below is correct only up to order $\rho^2$, and the terms of order $\rho^4$ should be dropped!!},
\begin{equation}
	\Sigma_m^n = \bar \lambda \left( 2(\rho^2 + \rho^4) \Sigma_a^{n-3} + (2 \rho^2 + \rho^4) \Sigma_m^{n-3} \right)
\end{equation}
Since $\Sigma_m^{n-3} + \Sigma_a^{n-3} = 1$, and knowing that $\Sigma_a^{n-3} = \bar \lambda$, we have finally
\begin{equation}
\boxed{
	\Sigma_m^n = \bar \lambda \left( 2 \rho^2 + \rho^4 + \bar \lambda \rho^4 \right)
}
\end{equation}
Finally,
\begin{equation}
\boxed{
	\bar \lambda(\rho) = \frac{1}{1+2\rho^2} +\mathcal{O}(\rho^2)
}
\end{equation}
\subsubsection{Molecular chains}

\section{The fractal dimensions}

\subsection{Definitions}

We distinguish between \emph{spectral dimensions} depending on the spectral index $a$, and \emph{local dimensions} depending on the spatial index $i$.

\textbf{Dimension of the spectrum} $\boxed{D_q}$, is given by the partition function
\begin{equation}
	\Gamma(\tau) = \lim_{n\rightarrow \infty} \sum_{a=1}^{F_n} \frac{(\mu^n(I_a))^q}{(\Delta_a^n)^\tau}
\end{equation}
where
\begin{equation}
	\d \mu^n(E) = \frac{1}{F_n}\sum_a \delta(E-E_a) \d E
\end{equation}
or rather
\begin{equation}
	\d \mu^n(E) = \frac{1}{F_n}\sum_a \d{m_a^n}(E)
\end{equation}
where $m_a^n$ is the spectral (Lebesgue) measure on the interval $I_a^n = [E_a,E_a+\Delta_a^n[$.
Here, since the bands are non-degenerate, $m_a^n$ is normalized to unity: $m_a^n(I_b^n)=\delta_{a,b}$.

$\Gamma(\tau)$ is finite and nonzero for a single value of $\tau$, 
\begin{equation}
	\tau(q) = (q-1)D_q
\end{equation}

\textbf{Spectral dimension of the wavefunctions} $\boxed{D^x_q(a)}$ is given by
\begin{equation}
	D^x_q(a) = \lim_{n\rightarrow \infty} \frac{1}{q-1} \frac{\log \chi^n_q(a)}{\log F_n}
\end{equation}
where $\chi^n_q(a)$ is the weighted probability sum
\begin{equation}
	\chi^n_q(a) = \sum_{i=1}^{F_n} |\psi^n(i,a)|^{2q}
\end{equation}

\textbf{Averaged dimension of the wavefunctions} $\boxed{\overline{D^x_q}}$:
\begin{equation}
	\overline{\chi^n_q} = \frac{1}{F_n}\sum_a \sum_i |\psi^n(i,a)|^{2q}
\end{equation}

\textbf{Local dimension of the local spectral measure} $\boxed{D^\mu_q(i)}$
\begin{equation}
	\Gamma^\mu(\tau(i)) = \lim_n \sum_a \frac{\left(\mu^n(i,\Delta_a^n)\right)^q}{(\Delta_a^n)^\tau}
\end{equation}
where
\begin{equation}
	\d \mu^n(i,E) = \sum_a |\psi^n(i,a)|^2 \delta(E-E_a) \d E
\end{equation}
or rather
\begin{equation}
	\d \mu^n(i,E) = \sum_a |\psi^n(i,a)|^2 \d{m_a}(E)
\end{equation}
Note that $\d \mu^n(i,E) = \d \mu^n(E)$ for plane waves.

\textbf{Local dimension of the wavefunctions} $\boxed{\tilde D_q(i)}$
\begin{equation}
	\tilde \chi^n_q(i) = \sum_a|\psi^n(i,a)|^{2q}
\end{equation}

\subsection{Conjectures}
\begin{equation}
	\tilde{D}_q(i)  \simlim{\rho}{0} D^x_q(a)
\end{equation}
if $a=i$ (assuming we label energy bands by increasing energy, and that we label sites in the co-numerotation).
Clearly this does no longer hold when $\rho$ is finite.
However there may exist a value of $q$ for which the equality holds nonperturbatively.

\begin{equation}
	D^\mu_q(i) = \alpha(i) \tilde{D}_q(i)
\end{equation}
Actually, at least in the perturbative limit we find that \emph{spatial} fractal dimensions only depend on the fraction of (renormalization group) time spent in molecular/atomic clusters, and on the fraction of time we switch from molecular to atomic clusters, and not on the order in which we visit atomic/molecular clusters. \\
Similarly, \emph{spectral} fractal dimensions only depends on the fraction of time spent in lateral/central energy clusters, and on the fraction of time we switch from lateral to central clusters.

So, \textit{a priori} the spatial fractal dimensions depend on $n_m$, time spent on molecular clusters, $n_a$, time spent on atomic clusters and on $n_{ma}$, time spent switching from molecular to atomic clusters. 
This makes 3 variables, but they are not independant! Because when we renormalize to an atomic cluster the number of sites is reduced by a factor $F_{n-3}/F_n$, while when we renormalize to a molecular cluster the reduction factor is $F_{n-2}/F_n$, we have
\begin{equation}
	3 n_a + 2 n_m = n
\end{equation}
where $n$ is the total number of renormalization steps\footnote{This is only valid in the limit $n \rightarrow \infty$. For $n$ finite we actually have $3 n_a + 2 n_m = n \pm 1$.}.
So, in practice we only have to independant variables.

To encode these informations, for \emph{spatial} dimensions we let
\begin{equation}
	  \begin{cases}
        x = n_m/n \\
        y = n_{ma}/n
      \end{cases}
\end{equation} 
while for \emph{spectral} dimensions we let
\begin{equation}
	  \begin{cases}
        x = n_l/n \\
        y = n_{lc}/n
      \end{cases}
\end{equation} 

In the perturbative limit, we thus have the slightly more general relation
\begin{equation}
	D^\mu_q(x,y) = \alpha(x) \tilde{D}_q(x,y)
\end{equation}
where $x$ is a spatial label for spatial dimensions, and a spectral label for $\alpha$.

\section{Asymptotic results}

\textbf{Averaged dimension of the wavefunctions} $\boxed{\overline{D^x_q}}$:

\begin{equation}
	\tilde{D}_q(x,y) \log \om = \frac{q}{q-1} \left( x \log \lb + \frac{1-2x}{3} \log \lbbar \right) - \frac{1}{q-1}\left( (x-y) \log\left(\frac{1}{2 c_q}\right) + y \log\left(\frac{1}{2 s_q}\right) \right)
\end{equation} 
since $\lb, \lbbar, c_q, s_q$ are only known up to second order in $\rho$, we also have
\begin{equation}
	\tilde{D}_q(x,y) \log \om = x \log\left( \frac{1}{2} \right) -\frac{q}{q-1} \frac{4-5x}{6} \rho^2 -\frac{1}{q-1}(-x+2y) \rho^{2q}
\end{equation}
\footnote{In particular, we have $\tilde{D}_q(x,y) \simlim{\rho}{0} x\frac{\log 2}{\log \tau}$. This makes perfect sense: in that limit only isolated molecules remain, and on each molecular site we have presence probability $1/(2 F_n)$.}


\bibliography{perturbative_fractal_dimensions.bib}{}
\bibliographystyle{plain}
\end{document}