\documentclass[11pt]{article}

\usepackage[utf8]{inputenc}
\usepackage[T1]{fontenc}
\usepackage[english, french]{babel} %français
\usepackage{amsmath}
\usepackage{amsfonts}
\usepackage{makeidx}
\usepackage{graphicx}
\usepackage[left=2cm,right=2cm,top=2cm,bottom=2cm]{geometry}
\usepackage{mathtools} %dcases
\usepackage{braket} %quantum mechanics
\usepackage[colorlinks=true, linkcolor=black, citecolor=black]{hyperref} % hyperlinks
\usepackage{tikz} % drawing in LaTeX

% the equal sign I use to define something
\newcommand{\define}{\ensuremath{ \overset{\text{def}}{=} }}

% differential element
\renewcommand{\d}[1]{\mathrm{d}#1}

\title{\textbf{Fractal dimensions of the Fibonacci chain \emph{via} a perturbative renormalization group.}}
\author{Nicolas Macé}
\date{9 décembre 2014}
\begin{document}

\selectlanguage{english}

\maketitle

\section{The renormalization equations}

\subsection{Threefold decimation}

\begin{center}
    	\begin{tikzpicture}[scale=1]
    		\newcommand{\orig}{-1.5}
    		\newcommand{\trans}{2}
    		\newcommand{\vertspac}{-2.}
    	
    		% initial chain
    	
    		% bonds 
        	\draw[-] (\orig, 0) -- (\orig+\trans, 0) node [midway, above] {$t_w$};
			\draw[-,double] (\orig+\trans,0) -- (\orig+2*\trans,0) node [midway, above] {$t_s$};
			\draw[-] (\orig+2*\trans,0) -- (\orig+3*\trans,0) node [midway, above] {$t_w$};	
			\draw[-,double] (\orig+3*\trans,0) -- (\orig+4*\trans,0) node [midway, above] {$t_s$};
			\draw[-] (\orig+4*\trans,0) -- (\orig+5*\trans,0) node [midway, above] {$t_w$};
			\draw[-] (\orig+5*\trans,0) -- (\orig+6*\trans,0) node [midway, above] {$t_w$};
			\draw[-,double] (\orig+6*\trans,0) -- (\orig+7*\trans,0) node [midway, above] {$t_s$};
			\draw[-] (\orig+7*\trans,0) -- (\orig+8*\trans,0) node [midway, above] {$t_w$};
    	
    		% sites
			\foreach \x in {0,...,8}
		      \filldraw (\orig+\x*\trans,0) circle (0.05) node [below] {$\ket{\x}$};;
		      
		    % atomic chain
        	\draw[-] (\orig, \vertspac) -- (\orig+\trans, \vertspac) node [midway, above] {$t_w$};		    
		    

		\end{tikzpicture}
\end{center}

\subsection{Renormalization of the couplings}

\subsubsection{Atomic chain}

In \cite{Zhong1991} one can find an exact renormalization group transformation linking a large chain (density of sites 1), to three smaller chains:
\begin{itemize}
	\item a chain of atomic sites (density of sites $1/\tau^3$)
	\item two chains of molecular sites (density of sites $1/\tau^2$).
\end{itemize}

In view of passing to a perturbative framework, we define
\begin{equation}
	\rho \define \frac{t_w}{t_s}
\end{equation}
A renormalization group operation generically translates the spectrum (because the one-sites energy are all renormalized by the same amount), and rescales it (because the hopping energies are all renormalized by the samed amount).
To take into account these effects, it is a good idea to look at the spectrum in the ``co-moving frame'':
\begin{equation}
	x_{\alpha}(E) \define \frac{E - V_{\alpha}}{t_{\alpha}}
\end{equation}
where $\alpha \in \{ s, w \}$.

The renormalization equations given in \cite{Zhong1991} rewrite
\begin{align}
	\bar{x}_w = x_w + 2 \rho^2 \frac{1}{x_s(1-x_s^2)}
\end{align}

\subsubsection{Molecular chains}

\subsection{Renormalization of the wavefunctions}

\subsubsection{Atomic chain}

\begin{center}
    	\begin{tikzpicture}[scale=1]
    		\newcommand{\orig}{-1.5}
    		\newcommand{\trans}{2}
    	
    		% bonds 
        	\draw[-] (\orig, 0) -- (\orig+\trans, 0) node [midway, above] {$t_0$};
			\draw[-,double] (\orig+\trans,0) -- (\orig+2*\trans,0) node [midway, above] {$t_1$};
			\draw[-] (\orig+2*\trans,0) -- (\orig+3*\trans,0) node [midway, above] {$t_0$};			
    	
    		% sites
			\filldraw (\orig,0) circle (0.05) node [below] {$\ket{0}$};
			\filldraw (\orig+\trans,0) circle (0.05) node [below] {$\ket{1}$};
			\filldraw (\orig+2*\trans,0) circle (0.05) node [below] {$\ket{2}$};
			\filldraw (\orig+3*\trans,0) circle (0.05) node [below] {$\ket{3}$};

		\end{tikzpicture}
\end{center}

\subsubsection{Molecular chains}

\section{The fractal dimensions}

\bibliography{perturbative_fractal_dimensions.bib}{}
\bibliographystyle{plain}
\end{document}