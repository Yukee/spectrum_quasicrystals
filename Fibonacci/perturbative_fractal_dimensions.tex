\documentclass[11pt]{article}

\usepackage[utf8]{inputenc}
\usepackage[T1]{fontenc}
\usepackage[english, french]{babel} %français
\usepackage{amsmath}
\usepackage{amsfonts}
\usepackage{makeidx}
\usepackage{graphicx}
\usepackage[left=2cm,right=2cm,top=2cm,bottom=2cm]{geometry}
\usepackage{mathtools} %dcases
\usepackage{braket} %quantum mechanics
\usepackage[colorlinks=true, linkcolor=black, citecolor=black]{hyperref} % hyperlinks
\usepackage{tikz} % drawing in LaTeX
\usepackage{ dsfont } % hollow letters

% the equal sign I use to define something
\newcommand{\define}{\ensuremath{ \overset{\text{def}}{=} }}

% differential element
\renewcommand{\d}[1]{\mathrm{d}#1}

\title{\textbf{Fractal dimensions of the Fibonacci chain \emph{via} a perturbative renormalization group.}}
\author{Nicolas Macé}
\date{9 décembre 2014}
\begin{document}

\selectlanguage{english}

\maketitle

\section{Constructing the Fibonacci chain: the alphabet of couplings.}

\subsection{The Fibonacci sequence and its statistical properties}

We consider the following inflation rule on the 2 letters alphabet $\mathcal{A} = \{t_s, t_w \}$:
\begin{equation}
	r \define \begin{cases}
        t_{w} & \rightarrow t_w t_s \\
        t_s & \rightarrow t_w
      \end{cases}
\end{equation} 
Starting from the seed $t_s$, we define the $n^\text{th}$ \emph{Fibonacci word} (aka \emph{rabbit word}):
\begin{equation}
	S_n \define r^{n}(t_s)
\end{equation}
One can show the $S_n$ converges\footnote{For the product topology on $\mathcal{A}^\mathds{N}$ induced by the discrete topology on $\mathcal{A}$.}. We call $S_\infty = \lim_{n \rightarrow \infty} S_n$ the \emph{Fibonacci sequence}.

\textbf{Statistical properties of the Fibonacci sequence}


The inflation rule $r$ has the substitution matrix
\begin{equation}
	\text{Sub}(r) = 
	\begin{bmatrix}
	0 & 1\\
	1 & 1\\
\end{bmatrix}
\end{equation}
The eigenvalues of the substitution matrix are $\tau$ and $\omega \define -1/\tau$, where $\tau = (1+\sqrt{5})/2$ is the golden ratio.
There are thus $\tau$ times more $t_w$ than $t_s$ in $S_\infty$.
More precisely, we have 
\begin{equation}
	\text{Sub}(r) = \mathcal{R}(\theta) 
	\begin{bmatrix}
	\tau & 0\\
	0 & \omega \\
	\end{bmatrix}
	\mathcal{R}(\theta)^{-1}
\end{equation}
where $\mathcal{R}(\theta)$ is the rotation by the angle $\theta$ such that
\begin{equation}
\begin{cases}
\cos \theta = 1/\sqrt{2+\tau} \\
\sin \theta = 1/\sqrt{2+\omega}
\end{cases}
\end{equation}
Knowing this we can easily compute $\text{Sub}(r^n) = \text{Sub}(r)^n$. We immediately deduce the frequency of letters in $S_n$:
\begin{equation}
	\begin{cases}
	\#\{ t_s \in S_n \}  & = F_{n-2} \\
	\#\{ t_w \in S_n \} & = F_{n-1}
	\end{cases}
\end{equation}
where $F_n$ is the $n^\text{th}$ \emph{Fibonacci number} ($F_0 = 1$, $F_1 = 1$, $F_2 = 2$, ...).

\subsection{Deflation rules; atoms and molecules.}

Consider the slightly different inflation rule
\begin{equation}
	\tilde{r} = \begin{cases}
        t_{w} & \rightarrow t_s t_w \\
        t_s & \rightarrow t_w
      \end{cases}
\end{equation} 
It has the same substitution matrix as $r$, and therefore the words produced by this rule have the same frequency of letters as the Fibonacci words. Furthermore they are locally undistinguishable from the Fibonacci words\footnote{Both word sequences admit the same 3 local environments: $t_w t_s$, $t_s t_w$ and $t_w t_w$.}. 
Any combination of these two substitution rules (eg $r \tilde{r} r r \tilde{r}...$) has again the same properties. We will also call words produced by such sequences Fibonacci words. 

\textbf{Molecules} \\

\begin{figure}[htp]
\centering
    	\begin{tikzpicture}[scale=1]
    		\newcommand{\orig}{-1.5}
    		\newcommand{\trans}{1.5}
    		\newcommand{\vertspac}{-2.}
    	
    		% initial chain
    	
    		% bonds 
        	\draw[-] (\orig, 0)  node [left] {$F_{n} (8)$}  -- (\orig+\trans, 0);
			\draw[-,double] (\orig+\trans,0) -- (\orig+2*\trans,0); % node [midway, above] {$t_s$};
			\draw[-] (\orig+2*\trans,0) -- (\orig+3*\trans,0); % node [midway, above] {$t_w$};	
			\draw[-,double] (\orig+3*\trans,0) -- (\orig+4*\trans,0); % node [midway, above] {$t_s$};
			\draw[-] (\orig+4*\trans,0) -- (\orig+5*\trans,0); % node [midway, above] {$t_w$};
			\draw[-] (\orig+5*\trans,0) -- (\orig+6*\trans,0); % node [midway, above] {$t_w$};
			\draw[-,double] (\orig+6*\trans,0) -- (\orig+7*\trans,0); % node [midway, above] {$t_s$};
			\draw[-] (\orig+7*\trans,0) -- (\orig+8*\trans,0); % node [midway, above] {$t_w$};
    	
    		% sites
			\foreach \x in {0,...,7}
		      \filldraw (\orig+\x*\trans,0) circle (0.05); % node [below] {$\ket{\x}$};
		      
			% molecular chains
			
			\foreach \x in {1}
			{
				\draw[-] (\orig, \x*\vertspac) node [left] {$F_{n-2} (3)$} -- (\orig+1.5*\trans, \x*\vertspac);
				\draw[-,double] (\orig+1.5*\trans, \x*\vertspac) -- (\orig+3.5*\trans, \x*\vertspac);
				\draw[-] (\orig+3.5*\trans, \x*\vertspac) -- (\orig+6.5*\trans, \x*\vertspac);
				\draw[-,double] (\orig+6.5*\trans, \x*\vertspac) -- (\orig+8*\trans, \x*\vertspac);
				
				\filldraw (\orig+1.5*\trans,\x*\vertspac) circle (0.05);
				\filldraw (\orig+3.5*\trans,\x*\vertspac) circle (0.05);
				\filldraw (\orig+6.5*\trans,\x*\vertspac) circle (0.05);
			}
		\end{tikzpicture}
\caption{The molecular deflation rule $d_m$ illustrated on a length 8 Fibonacci word.}
\label{fig:mol_defl}
\end{figure}

We have
\begin{equation}
	\tilde{r} r = \begin{cases}
        t_{w} & \rightarrow t_s t_w t_w \\
        t_s & \rightarrow t_s t_w
      \end{cases}
\end{equation} 
Which admits as an inverse the \emph{deflation rule}
\begin{equation}
	d_m = \begin{cases}
        t_{w} & \leftarrow t_s t_w t_w \\
        t_s & \leftarrow t_s t_w
      \end{cases}
\end{equation}
This deflation rule reduces a word of length $F_n$ to word of length $F_{n-2}$ (fig. \eqref{fig:mol_defl}).

\textbf{Atoms} \\

\begin{figure}[htp]
\centering
    	\begin{tikzpicture}[scale=1]
    		\newcommand{\orig}{-1.5}
    		\newcommand{\trans}{1.5}
    		\newcommand{\vertspac}{-2.}
    	
    		% initial chain
    	
    		% bonds 
        	\draw[-] (\orig, 0)  node [left] {$F_{n} (8)$}  -- (\orig+\trans, 0);
			\draw[-,double] (\orig+\trans,0) -- (\orig+2*\trans,0); % node [midway, above] {$t_s$};
			\draw[-] (\orig+2*\trans,0) -- (\orig+3*\trans,0); % node [midway, above] {$t_w$};	
			\draw[-,double] (\orig+3*\trans,0) -- (\orig+4*\trans,0); % node [midway, above] {$t_s$};
			\draw[-] (\orig+4*\trans,0) -- (\orig+5*\trans,0); % node [midway, above] {$t_w$};
			\draw[-] (\orig+5*\trans,0) -- (\orig+6*\trans,0); % node [midway, above] {$t_w$};
			\draw[-,double] (\orig+6*\trans,0) -- (\orig+7*\trans,0); % node [midway, above] {$t_s$};
			\draw[-] (\orig+7*\trans,0) -- (\orig+8*\trans,0); % node [midway, above] {$t_w$};
    	
    		% sites
			\foreach \x in {0,...,7}
		      \filldraw (\orig+\x*\trans,0) circle (0.05); % node [below] {$\ket{\x}$};
		      
		    % atomic chain
		    
        	\draw[-] (\orig, \vertspac)  node [left] {$F_{n-3} (2)$}  -- (\orig+5*\trans, \vertspac);
			\draw[-,double] (\orig+5*\trans,\vertspac) -- (\orig+8*\trans,\vertspac); % node [midway, above] {$t_s$};
			
			\filldraw (\orig,\vertspac) circle (0.05); % node [below] {$\ket{\x}$};
			\filldraw (\orig+5*\trans,\vertspac) circle (0.05); % node [below] {$\ket{\x}$};
%			\filldraw (\orig+8*\trans,\vertspac) circle (0.05); % node [below] {$\ket{\x}$};
		\end{tikzpicture}
\caption{The atomic deflation rule $d_a$ illustrated on a length 8 Fibonacci word.}
\end{figure}

\subsection{The Fibonacci Hamiltonian and its approximants}

There is a natural quantum system associated to $S_n$. 
It is described by the tight-binding Hamiltonian $H_n$, whose sequence of couplings is given by $S_n$.

\textbf{Boundary contiditons:} as there are $F_n$ words in $S_n$, $H_n$ has $F_n$ couplings. These $F_n$ couplings are jump amplitudes between two neighbouring atomic sites. There are thus \textit{a priori} $F_n + 1$ atomic sites, but we are going identify the first and the last atomic sites, ie use \textit{periodic boundary conditions}.
See eq. \eqref{eq:h8} for the example of the sixth Hamiltonian, with $F_6 = 8$ atomis sites.

\begin{equation}
\label{eq:h8} 
	H_6 = 
	\begin{bmatrix}
	0 & t_w &   &   &   &   &   & t_w\\
	t_w & 0 & t_s &   &   &   &   &  \\
	  & t_s & 0 & t_w &   &   &   &  \\
	  &   & t_w & 0 & t_s &   &   &  \\
	  &   &   & t_s & 0 & t_w &   &  \\
	  &   &   &   & t_w & 0 & t_w &  \\
	  &   &   &   &   & t_w & 0 & t_s\\
	t_w &   &   &   &   &   & t_s & 0\\
\end{bmatrix}
\end{equation}

Tha Hamiltonian $H_n$ is called the \textit{$n^\text{th}$ approximant}. $H_\infty = \lim_{n \rightarrow \infty} H_n$ is called the \textit{Fibonacci Hamiltonian}.

\section{The renormalization equations}

\subsection{Threefold decimation}
We wish to consider the sequence of approximants in the \textit{strong quasiperiodicity limit}: $t_s/t_w \ll 1$.
In view of passing to this perturbative framework, we define the small parameter
\begin{equation}
	\rho \define \frac{t_w}{t_s}
\end{equation}

\textbf{At leading order} in $\rho$ weak bonds disappear: $t_w = 0$. 
Therefore we can diagonalize separately \textit{molecules}, ie atomic sites linked by a strong coupling, and \textit{atoms}, ie atomic sites surrounded by two weak couplings.

\textbf{At next-to-leading order}, two neighbouring molecules/atoms can interact, and the study of the chain $S_n$ is naturally equivalent to the study of an atomic chain $d_a(S_n) = S_{n-3}$ and two molecular chains $d_m(S_n) = S_{n-2}$. 

The \textit{deflation} of the chain $S_n$ translates into a \textit{renormalization} of the Hamiltonian $H_n$: $H_n$ is related to $H_{n-2}$ by the decimation of atomic sites, and to $H_{n-3}$ by the decimation of molecular sites. 
Though the action of the deflation rules $d_a$ and $d_m$ on $S_n$ is simple, the action of the decimation rules on $H_n$ is far from trivial.
In particular, after a decimation, couplings and wavefunctions are renormalized. We are now interested in characterizing this renormalization.

\subsection{Renormalization of the couplings}

\subsubsection{Atomic chain}

In \cite{Zhong1991} one can find an exact renormalization group transformation linking a large chain (density of sites 1), to three smaller chains:
\begin{itemize}
	\item a chain of atomic sites (density of sites $1/\tau^3$)
	\item two chains of molecular sites (density of sites $1/\tau^2$).
\end{itemize}

In view of passing to a perturbative framework, we define
\begin{equation}
	\rho \define \frac{t_w}{t_s}
\end{equation}
A renormalization group operation generically translates the spectrum (because the one-sites energy are all renormalized by the same amount), and rescales it (because the hopping energies are all renormalized by the samed amount).
To take into account these effects, it is a good idea to look at the spectrum in the ``co-moving frame'':
\begin{equation}
	x_{\alpha}(E) \define \frac{E - V_{\alpha}}{t_{\alpha}}
\end{equation}
where $\alpha \in \{ s, w \}$.

The renormalization equations given in \cite{Zhong1991} rewrite
\begin{align}
	\bar{x}_w = x_w + 2 \rho^2 \frac{1}{x_s(1-x_s^2)}
\end{align}

\subsubsection{Molecular chains}

\subsection{Renormalization of the wavefunctions}

\subsubsection{Atomic chain}

\begin{center}
    	\begin{tikzpicture}[scale=1]
    		\newcommand{\orig}{-1.5}
    		\newcommand{\trans}{2}
    	
    		% bonds 
        	\draw[-] (\orig, 0) -- (\orig+\trans, 0) node [midway, above] {$t_0$};
			\draw[-,double] (\orig+\trans,0) -- (\orig+2*\trans,0) node [midway, above] {$t_1$};
			\draw[-] (\orig+2*\trans,0) -- (\orig+3*\trans,0) node [midway, above] {$t_0$};			
    	
    		% sites
			\filldraw (\orig,0) circle (0.05) node [below] {$\ket{0}$};
			\filldraw (\orig+\trans,0) circle (0.05) node [below] {$\ket{1}$};
			\filldraw (\orig+2*\trans,0) circle (0.05) node [below] {$\ket{2}$};
			\filldraw (\orig+3*\trans,0) circle (0.05) node [below] {$\ket{3}$};

		\end{tikzpicture}
\end{center}

At leading order in $\rho$, an atomic wavefunction of energy $E$ is sent by $d_a$ to an atomic wavefunction of energy $\bar z E$.
If $i$ is the label of an atomic site of the chain $S_n$, we have
\begin{equation}
	\psi_i(E)  = \beta_i\left(\{ \psi_{i'}(\bar z E) \}_{i'}\right)
\end{equation} 
We assume that the transformation $\beta$ is linear:
\begin{equation}
	\psi_i(E)  = \beta_{ii'}  \psi_{i'}(\bar z E)
\end{equation}
and even more than that, that it is a simple homothety: $\beta_{ii'} = \sqrt{\lambda} \delta_{d_a(i)i'}$.
We therefore have
\begin{equation}
\label{eq:renorm_a}
	\psi_i(E) = \sqrt{\lambda} \psi_{d_a(i)}(\bar z E)
\end{equation}
$\lambda$ is constrained by the unitarity of the wavefunction:
\begin{equation}
	\Sigma^n \define \sum_{i=1}^{F_n} |\psi_i|^2 = 1
\end{equation}
Let us note that we can partition the set of atomic sites of $S_n$ into the set of atomic sites that remain atomic sites after deflation $S_{a'}$ and the set atomic sites that become molecular sites after deflation $S_{m'}$.
Using this partition to transform the sum on all sites, we get
\begin{equation}
	\sum_{i=1}^{F_n} |\psi_i|^2 = \sum_{i \in S_{a'}} \left( |\psi_i|^2 + |\psi_{i+2}|^2 + |\psi_{i-2}|^2 + |\psi_{i+4}|^2 + |\psi_{i-4}|^2 \right) + \sum_{i \in S_{m'}} \left( |\psi_i|^2 + |\psi_{i+2}|^2 + |\psi_{i-2}|^2 + |\psi_{i\pm4}|^2 \right)
\end{equation}
We can also partition this sum into a sum over atomic sites $\Sigma^n_a$ and over molecular sites $\Sigma^n_m$.
We have
\begin{equation}
	\begin{cases}
	\Sigma^n_a &= \sum_{i \in S_{a'}}  |\psi_i|^2  + \sum_{i \in S_{m'}}  |\psi_i|^2\\
	\Sigma^n_m &= \sum_{i \in S_{a'}} \left( |\psi_{i+2}|^2 + |\psi_{i-2}|^2 + |\psi_{i+4}|^2 + |\psi_{i-4}|^2 \right) + \sum_{i \in S_{m'}} \left(|\psi_{i+2}|^2 + |\psi_{i-2}|^2 + |\psi_{i\pm4}|^2 \right)
	\end{cases}
\end{equation}
Now, beacause of the way we assumed the wavefunction on atomic sites is renormalized (eq. \eqref{eq:renorm_a}), we have
\begin{align}
	\Sigma_a^n &= \bar \lambda  \left(  \sum_{i \in S_{a'}}  |\psi_{d_a(i)}|^2  + \sum_{i \in S_{m'}}  |\psi_{d_a(i)}|^2 \right) \\
	&= \bar \lambda \sum_{i = 1}^{F_{n-3}} |\psi_i|^2 \\
	&= \bar \lambda \Sigma^{n-3}
\end{align}
and thus
\begin{equation}
\label{eq:renorm_aa}
\boxed{
	\Sigma_a^n = \bar \lambda
	}
\end{equation}

Now, in the perturbative limit, the great thing is that the wafunction at molecular sites can be expressed in terms of the wavefunction at atomic sites\footnote{To derive these relations, we use the fact that $E$ being an atomic energy, it is at most of order $\rho$. We also use the fact that at leading order the wavefunction is of the same order on all atomic sites.}! 
\begin{equation}
	\begin{cases}
	\psi_{i\pm2} &= - \rho \psi_i + \mathcal{O}(\rho E \psi_i) \\
	\psi_{i\pm4} &= \phantom{-}\rho^2 \psi_i  + \mathcal{O}(\rho^4 \psi_i)
	\end{cases}
\end{equation}
Using this together with \eqref{eq:renorm_a} we get\footnote{If energy really is at most of order $\rho$, the formula below is correct only up to order $\rho^2$, and the terms of order $\rho^4$ should be dropped!!},
\begin{equation}
	\Sigma_m^n = \bar \lambda \left( 2(\rho^2 + \rho^4) \Sigma_a^{n-3} + (2 \rho^2 + \rho^4) \Sigma_m^{n-3} \right)
\end{equation}
Since $\Sigma_m^{n-3} + \Sigma_a^{n-3} = 1$, and knowing that $\Sigma_a^{n-3} = \bar \lambda$, we have finally
\begin{equation}
\boxed{
	\Sigma_m^n = \bar \lambda \left( 2 \rho^2 + \rho^4 + \bar \lambda \rho^4 \right)
}
\end{equation}
Finally,
\begin{equation}
\boxed{
	\bar \lambda(\rho) = \frac{1}{1+2\rho^2} +\mathcal{O}(\rho^2)
}
\end{equation}
\subsubsection{Molecular chains}

\section{The fractal dimensions}

\bibliography{perturbative_fractal_dimensions.bib}{}
\bibliographystyle{plain}
\end{document}