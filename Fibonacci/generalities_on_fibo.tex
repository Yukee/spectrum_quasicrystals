\documentclass[11pt]{article}

\usepackage[utf8]{inputenc}
\usepackage[T1]{fontenc}
\usepackage[english, french]{babel} %français
\usepackage{amsmath}
\usepackage{amsfonts}
\usepackage{makeidx}
\usepackage{graphicx}
\usepackage[left=2cm,right=2cm,top=2cm,bottom=2cm]{geometry}
\usepackage{mathtools} %dcases
\usepackage{braket} %quantum mechanics
\usepackage[colorlinks=true, linkcolor=black, citecolor=black]{hyperref} % hyperlinks
\usepackage{tikz} % drawing in LaTeX
\usepackage{ dsfont } % hollow letters
% the equal sign I use to define something
\newcommand{\define}{\ensuremath{ \overset{\text{def}}{=} }}

% differential element
\renewcommand{\d}[1]{\mathrm{d}#1}

% similar symbol with a limit underneath
\newcommand{\simlim}[2]{\ensuremath{ \underset{#1 \rightarrow #2}{\sim} }}

\newcommand{\om}{\ensuremath{\omega}}
\newcommand{\lb}{\ensuremath{\lambda}}
\newcommand{\lbbar}{\ensuremath{\overline{\lambda}}}

\title{\textbf{Generalities on the Fibonacci Hamiltonian.}}
\author{Nicolas Macé}
\date{9 décembre 2014}
\begin{document}

\selectlanguage{english}

\maketitle

\section{Constructing the Fibonacci chain: the alphabet of couplings.}

\subsection{The Fibonacci sequence and its statistical properties}

We consider the following inflation rule on the 2 letters alphabet $\mathcal{A} = \{t_s, t_w \}$:
\begin{equation}
	r \define \begin{cases}
        t_{w} & \rightarrow t_w t_s \\
        t_s & \rightarrow t_w
      \end{cases}
\end{equation} 
Starting from the seed $t_s$, we define the $n^\text{th}$ \emph{Fibonacci word} (aka \emph{rabbit word}):
\begin{equation}
	S_n \define r^{n}(t_s)
\end{equation}
One can show the $S_n$ converges\footnote{For the product topology on $\mathcal{A}^\mathds{N}$ induced by the discrete topology on $\mathcal{A}$.}. We call $S_\infty = \lim_{n \rightarrow \infty} S_n$ the \emph{Fibonacci sequence}.

\textbf{Statistical properties of the Fibonacci sequence}


The inflation rule $r$ has the substitution matrix
\begin{equation}
	\text{Sub}(r) = 
	\begin{bmatrix}
	0 & 1\\
	1 & 1\\
\end{bmatrix}
\end{equation}
The eigenvalues of the substitution matrix are $\tau$ and $\omega \define -1/\tau$, where $\tau = (1+\sqrt{5})/2$ is the golden ratio.
There are thus $\tau$ times more $t_w$ than $t_s$ in $S_\infty$.
More precisely, we have 
\begin{equation}
	\text{Sub}(r) = \mathcal{R}(\theta) 
	\begin{bmatrix}
	\tau & 0\\
	0 & \omega \\
	\end{bmatrix}
	\mathcal{R}(\theta)^{-1}
\end{equation}
where $\mathcal{R}(\theta)$ is the rotation by the angle $\theta$ such that
\begin{equation}
\begin{cases}
\cos \theta = 1/\sqrt{2+\tau} \\
\sin \theta = 1/\sqrt{2+\omega}
\end{cases}
\end{equation}
Knowing this we can easily compute $\text{Sub}(r^n) = \text{Sub}(r)^n$. We immediately deduce the frequency of letters in $S_n$:
\begin{equation}
	\begin{cases}
	\#\{ t_s \in S_n \}  & = F_{n-2} \\
	\#\{ t_w \in S_n \} & = F_{n-1}
	\end{cases}
\end{equation}
where $F_n$ is the $n^\text{th}$ \emph{Fibonacci number} ($F_0 = 1$, $F_1 = 1$, $F_2 = 2$, ...).

\subsection{Deflation rules; atoms and molecules.}

Consider the slightly different inflation rule
\begin{equation}
	\tilde{r} = \begin{cases}
        t_{w} & \rightarrow t_s t_w \\
        t_s & \rightarrow t_w
      \end{cases}
\end{equation} 
It has the same substitution matrix as $r$, and therefore the words produced by this rule have the same frequency of letters as the Fibonacci words. Furthermore they are locally undistinguishable from the Fibonacci words\footnote{Both word sequences admit the same 3 local environments: $t_w t_s$, $t_s t_w$ and $t_w t_w$.}. 
Any combination of these two substitution rules (eg $r \tilde{r} r r \tilde{r}...$) has again the same properties. We will also call words produced by such sequences Fibonacci words. 

\textbf{Molecules} \\

\begin{figure}[htp]
\centering
    	\begin{tikzpicture}[scale=1]
    		\newcommand{\orig}{-1.5}
    		\newcommand{\trans}{1.5}
    		\newcommand{\vertspac}{-2.}
    	
    		% initial chain
    	
    		% bonds 
        	\draw[-] (\orig, 0)  node [left] {$F_{n} (8)$}  -- (\orig+\trans, 0);
			\draw[-,double] (\orig+\trans,0) -- (\orig+2*\trans,0); % node [midway, above] {$t_s$};
			\draw[-] (\orig+2*\trans,0) -- (\orig+3*\trans,0); % node [midway, above] {$t_w$};	
			\draw[-,double] (\orig+3*\trans,0) -- (\orig+4*\trans,0); % node [midway, above] {$t_s$};
			\draw[-] (\orig+4*\trans,0) -- (\orig+5*\trans,0); % node [midway, above] {$t_w$};
			\draw[-] (\orig+5*\trans,0) -- (\orig+6*\trans,0); % node [midway, above] {$t_w$};
			\draw[-,double] (\orig+6*\trans,0) -- (\orig+7*\trans,0); % node [midway, above] {$t_s$};
			\draw[-] (\orig+7*\trans,0) -- (\orig+8*\trans,0); % node [midway, above] {$t_w$};
    	
    		% sites
			\foreach \x in {0,...,7}
		      \filldraw (\orig+\x*\trans,0) circle (0.05); % node [below] {$\ket{\x}$};
		      
			% molecular chains
			
			\foreach \x in {1}
			{
				\draw[-] (\orig, \x*\vertspac) node [left] {$F_{n-2} (3)$} -- (\orig+1.5*\trans, \x*\vertspac);
				\draw[-,double] (\orig+1.5*\trans, \x*\vertspac) -- (\orig+3.5*\trans, \x*\vertspac);
				\draw[-] (\orig+3.5*\trans, \x*\vertspac) -- (\orig+6.5*\trans, \x*\vertspac);
				\draw[-,double] (\orig+6.5*\trans, \x*\vertspac) -- (\orig+8*\trans, \x*\vertspac);
				
				\filldraw (\orig+1.5*\trans,\x*\vertspac) circle (0.05);
				\filldraw (\orig+3.5*\trans,\x*\vertspac) circle (0.05);
				\filldraw (\orig+6.5*\trans,\x*\vertspac) circle (0.05);
			}
		\end{tikzpicture}
\caption{The molecular deflation rule $d_m$ illustrated on a length 8 Fibonacci word.}
\label{fig:mol_defl}
\end{figure}

We have
\begin{equation}
	\tilde{r} r = \begin{cases}
        t_{w} & \rightarrow t_s t_w t_w \\
        t_s & \rightarrow t_s t_w
      \end{cases}
\end{equation} 
Which admits as an inverse the \emph{deflation rule}
\begin{equation}
	d_m = \begin{cases}
        t_{w} & \leftarrow t_s t_w t_w \\
        t_s & \leftarrow t_s t_w
      \end{cases}
\end{equation}
This deflation rule reduces a word of length $F_n$ to word of length $F_{n-2}$ (fig. \eqref{fig:mol_defl}).

\textbf{Atoms} \\

\begin{figure}[htp]
\centering
    	\begin{tikzpicture}[scale=1]
    		\newcommand{\orig}{-1.5}
    		\newcommand{\trans}{1.5}
    		\newcommand{\vertspac}{-2.}
    	
    		% initial chain
    	
    		% bonds 
        	\draw[-] (\orig, 0)  node [left] {$F_{n} (8)$}  -- (\orig+\trans, 0);
			\draw[-,double] (\orig+\trans,0) -- (\orig+2*\trans,0); % node [midway, above] {$t_s$};
			\draw[-] (\orig+2*\trans,0) -- (\orig+3*\trans,0); % node [midway, above] {$t_w$};	
			\draw[-,double] (\orig+3*\trans,0) -- (\orig+4*\trans,0); % node [midway, above] {$t_s$};
			\draw[-] (\orig+4*\trans,0) -- (\orig+5*\trans,0); % node [midway, above] {$t_w$};
			\draw[-] (\orig+5*\trans,0) -- (\orig+6*\trans,0); % node [midway, above] {$t_w$};
			\draw[-,double] (\orig+6*\trans,0) -- (\orig+7*\trans,0); % node [midway, above] {$t_s$};
			\draw[-] (\orig+7*\trans,0) -- (\orig+8*\trans,0); % node [midway, above] {$t_w$};
    	
    		% sites
			\foreach \x in {0,...,7}
		      \filldraw (\orig+\x*\trans,0) circle (0.05); % node [below] {$\ket{\x}$};
		      
		    % atomic chain
		    
        	\draw[-] (\orig, \vertspac)  node [left] {$F_{n-3} (2)$}  -- (\orig+5*\trans, \vertspac);
			\draw[-,double] (\orig+5*\trans,\vertspac) -- (\orig+8*\trans,\vertspac); % node [midway, above] {$t_s$};
			
			\filldraw (\orig,\vertspac) circle (0.05); % node [below] {$\ket{\x}$};
			\filldraw (\orig+5*\trans,\vertspac) circle (0.05); % node [below] {$\ket{\x}$};
%			\filldraw (\orig+8*\trans,\vertspac) circle (0.05); % node [below] {$\ket{\x}$};
		\end{tikzpicture}
\caption{The atomic deflation rule $d_a$ illustrated on a length 8 Fibonacci word.}
\end{figure}

\subsection{The Fibonacci Hamiltonian and its approximants}

There is a natural quantum system associated to $S_n$. 
It is described by the tight-binding Hamiltonian $H_n$, whose sequence of couplings is given by $S_n$.

\textbf{Boundary contiditons:} as there are $F_n$ words in $S_n$, $H_n$ has $F_n$ couplings. These $F_n$ couplings are jump amplitudes between two neighbouring atomic sites. There are thus \textit{a priori} $F_n + 1$ atomic sites, but we are going identify the first and the last atomic sites, ie use \textit{periodic boundary conditions}.
See eq. \eqref{eq:h8} for the example of the sixth Hamiltonian, with $F_6 = 8$ atomis sites.

\begin{equation}
\label{eq:h8} 
	H_6 = 
	\begin{bmatrix}
	0 & t_w &   &   &   &   &   & t_w\\
	t_w & 0 & t_s &   &   &   &   &  \\
	  & t_s & 0 & t_w &   &   &   &  \\
	  &   & t_w & 0 & t_s &   &   &  \\
	  &   &   & t_s & 0 & t_w &   &  \\
	  &   &   &   & t_w & 0 & t_w &  \\
	  &   &   &   &   & t_w & 0 & t_s\\
	t_w &   &   &   &   &   & t_s & 0\\
\end{bmatrix}
\end{equation}

Tha Hamiltonian $H_n$ is called the \textit{$n^\text{th}$ approximant}. $H_\infty = \lim_{n \rightarrow \infty} H_n$ is called the \textit{Fibonacci Hamiltonian}.

\end{document}