\documentclass[11pt]{article}

\usepackage[utf8]{inputenc}
\usepackage[T1]{fontenc}
\usepackage[english, french]{babel} %français
\usepackage{amsmath}
\usepackage{amsfonts}
\usepackage{makeidx}
\usepackage{graphicx}
\usepackage[left=2cm,right=2cm,top=2cm,bottom=2cm]{geometry}
\usepackage{mathtools} %dcases
\usepackage{braket} %quantum mechanics
\usepackage[colorlinks=true, linkcolor=black, citecolor=black]{hyperref} % hyperlinks
\usepackage{tikz} % drawing in LaTeX

% the equal sign I use to define something
\newcommand{\define}{\ensuremath{ \overset{\text{déf}}{=} }}

% differential element
\renewcommand{\d}[1]{\mathrm{d}#1}

\title{\textbf{Méthode des perturbations sur la chaîne hiérarchique}}
\author{Nicolas Macé}
\date{lundi 15 septembre 2014}
\begin{document}

\maketitle

\section{Perturbation aux premiers ordres}

On considère le Hamiltonien :
\begin{equation}
H = H_0 + \lambda H_1
\end{equation}
avec $\lambda$ un petit paramètre.

On considère $\ket{\psi}$, un vecteur propre du Hamiltonien complet.
On peut le développer en puissances de $\lambda$ :
\begin{equation}
	\ket{\psi} = \ket{\psi^0} + \lambda \ket{\psi^1} + \lambda^2 \ket{\psi^2} + ...
\end{equation}
Par ailleurs on peut faire agir $\ket{\psi}$ sur le Hamiltonien complet, et développer le résultat en puissances de $\lambda$ :
\begin{equation}
	H \ket{\psi} = \left( W^{(0)} + \lambda W^{(1)} + \lambda^2 W^{(2)} + ...  \right) \ket{\psi}
\end{equation}
En utilisant que $H = H_0 + \lambda H_1$ on obtient
\begin{align}
	H_0 \ket{\psi^0} &= W^{(0)} \ket{\psi^{0}} \\
	H_0 \ket{\psi^1} + H_1\ket{\psi^0} &= W^{(0)} \ket{\psi^{1}} + W^{(1)} \ket{\psi^0} \\
	H_0 \ket{\psi^2} + H_1\ket{\psi^1} &= W^{(0)} \ket{\psi^{2}} + W^{(1)} \ket{\psi^1} + W^{(2)} \ket{\psi^0} \\
\text{etc.}  
\end{align}

\textbf{Perturbation à l'ordre 0}\\
Soit $E_n$ une énergie propre du Hamiltonien non perturbé $H_0$. On appelle $\ket{n,r}$ le $r^\text{ième}$ vecteur propre associé à $E_n$. 
On suppose que $\ket{\psi^0} \rightarrow \ket{\psi^0_{n,q}}$ vit dans le sous-espace propre de $H_0$ associé à $E_n$.
Alors, la première des relations précédentes nous dit que 
\begin{equation}
	W^{(0)}_{n,q}  = E_n
\end{equation}
À cet ordre, les énergies propres de $H$ sont celles de $H_0$, et les fonctions propres aussi.

\textbf{Perturbation à l'ordre 1}\\
En prenant le produit scalaire de la deuxième relation avec $\ket{n,r}$, on obtient
\begin{equation}
\label{ordre1}
	\braket{ n,r | H_1| \psi^0_{n,q} } = W^{(1)}_{nq} \braket{ n,r | \psi^0_{nq} }
\end{equation}
par conséquent, $\ket{\psi^0_{n,q}}$ est un vecteur propre de la restriction de $H_1$ au sous-espace propre de $H_0$ d'énergie $E_n$, et $W^{(1)}_{n,q}$ est la valeur propre associée.

À cet ordre, les énergies propres sont donc les 
\begin{align*}
	\varepsilon_{n,q} &= W^{(0)}_{n,q} + \lambda W^{(1)}_{n,q} \\
	\varepsilon_{n,q} &= E_n + \lambda W^{(1)}_{n,q}
\end{align*}
et les vecteurs propres associés sont les $\ket{\psi^0_{n,q}}$, combinaisons linéaires des $\ket{n,r}$. Les coefficients de $\ket{\psi^0_{n,q}}$ ainsi que les $W^{(1)}_{n,q}$ sont déterminés par l'équation aux valeurs propres \eqref{ordre1} (aussi appelée équation séculaire, parce qu'elle a servi en premier à Lagrange et Laplace pour déterminer perturbativement l'effet à long terme des interactions gravitationnelles entre les planètes du système solaire dans le cadre de la mécanique newtonienne).

\textbf{Perturbation à l'ordre 2}\\
Soit $P_{n,q}$ le projecteur orthogonal à $\ket{\psi^0_{n,q}}$. On réutilise la deuxième relation pour obtenir
\begin{equation}
	P_{n,q} \ket{\psi^1_{n,q}} = (E_n - H_0)^{-1} P_{n,q} H_1 \ket{\psi^0_{n,q}}
\end{equation}
Comme le vecteur propre total $\ket{\psi_{n,q}}$ est de norme 1, $\Re \braket{\psi^0_{n,q} | \psi^1_{n,q}} = 0$. Par ailleurs, on peut choisir la phase de $\ket{\psi_{n,q}}$ de telle sorte que $\Im \braket{\psi^0_{n,q} | \psi^1_{n,q}} = 0$. 
Cela étant fait, $P_{n,q} \ket{\psi^1_{n,q}} = \ket{\psi^1_{n,q}}$ et
\begin{equation}
\label{ordre2}
	\ket{\psi^1_{n,q}} = (E_n - H_0)^{-1} P_{n,q} H_1 \ket{\psi^0_{n,q}}
\end{equation}\footnote{Remarquons que $(E_n - H_0)^{-1}$ n'est bien défini que dans le sous-espace orthogonal à $E_n$. Pour que la théorie de perturbation à l'ordre 2 fonctionne, il faut donc que $\forall n, q$, $P_{n,q} H_1 \ket{\psi^0_{n,q}} \notin E_n$, c'est-à-dire que $H_1$ envoie $\ket{\psi^0_{n,q}}$ sur lui-même, et sur le sous-espace orthogonal à $E_n$. S'il n'y a pas de dégénérescence, cette condition est automatiquement vérifiée. S'il y a une dégénérescence, il faut donc que $\forall q, \forall p \neq q, \braket{\psi_{np}^{(0)} | H_1 | \psi_{nq}^{(0)}}=0$. Or $\braket{\psi_{np}^{(0)} | H_1 | \psi_{nq}^{(0)}}=W^{(1)}_{n,q} \braket{\psi_{np}^{(0)} | \psi_{nq}^{(0)}}$. Il suffit donc que les $\ket{\psi_{nq}^{(0)}}$ soient orthogonaux entre eux pour que la condition soit vérifiée. Or, c'est toujours possible d'orthogonaliser ces vecteurs. Ouf !}
À cet ordre, le vecteur propre est donc
\begin{align*}
	\ket{\psi_{n,q}} &= \ket{\psi^0_{n,q}} + \lambda \ket{\psi^1_{n,q}}  \\
	\ket{\psi_{n,q}} &= \ket{\psi^0_{n,q}} + \lambda (E_n - H_0)^{-1} P_{n,q} H_1 \ket{\psi^0_{n,q}}
\end{align*}
À cet ordre les valeurs propres sont aussi modifées par l'ajout de $W^{(2)}_{n,q}$, que je ne vais pas calculer ici ! Si on voulait le faire, il faudrait prendre le produit scalaire par $\ket{n,r}$ de la troisième relation.

\section{Application à la chaîne hiérarchique}

\begin{center}
    	\begin{tikzpicture}[scale=1]
    		\newcommand{\orig}{-1.5}
    		\newcommand{\trans}{2}
    	
    		% bonds 
        	\draw[-] (\orig, 0) -- (\orig+\trans, 0) node [midway, above] {$t_0$};
			\draw[-,dashed] (\orig+\trans,0) -- (\orig+2*\trans,0) node [midway, above] {$t_1$};
			\draw[-] (\orig+2*\trans,0) -- (\orig+3*\trans,0) node [midway, above] {$t_0$};			
    	
    		% sites
			\filldraw (\orig,0) circle (0.05) node [below] {$\ket{0}$};
			\filldraw (\orig+\trans,0) circle (0.05) node [below] {$\ket{1}$};
			\filldraw (\orig+2*\trans,0) circle (0.05) node [below] {$\ket{2}$};
			\filldraw (\orig+3*\trans,0) circle (0.05) node [below] {$\ket{3}$};

		\end{tikzpicture}
\end{center}

On écrit
\begin{equation}
	H = H_0 + H_1
\end{equation}
où $H_0$ contient les termes de saut sur les sites liés par le couplage $t_0$, et $H_1$ les termes de saut sur les sites liés par le couplage $t_1$. 
Ici, le petit paramètre est $\lambda = t_1/t_0$.

\textbf{Perturbation à l'ordre 0}\\
Les états propres de $H_0$ sont
\begin{align*}
	\ket{0,\pm} = \frac{1}{\sqrt{2}} \left( \ket{0} \pm \ket{1} \right) \\
	\ket{2,\pm} = \frac{1}{\sqrt{2}} \left( \ket{2} \pm \ket{3} \right)
\end{align*}
Les états $\ket{i,+}$ sont associés à l'énergie $t_0$, tandis que les états $\ket{i,-}$ sont associés à l'énergie $-t_0$.

\textbf{Perturbation à l'ordre 1}\\
On cherche maintenant les valeurs propres des restrictions $Q_{\pm} H_1 Q_{\pm}$ de $H_1$ aux deux sous-espaces propres de $H_0$, associés à $\pm t_0$.

On a
\begin{align*}
	H^{\pm}_{\text{eq}} &= H_0 + Q_\pm H_1 Q_\pm \\
	&= H_0 + \frac{t_1}{2} \left( \ket{0,\pm}\bra{1,\pm} + \ket{1,\pm}\bra{0,\pm} \right)
\end{align*}
Les états propres que l'on va déterminer seront ceux que l'on avait noté $\ket{\psi^0_{n,q}}$ précédemment. Ici, $n=\pm$ et $q=\pm$. On a
\begin{align*}
	\ket{\psi^0_{++}} = \frac{1}{\sqrt{2}} \left( \ket{0,+} + \ket{1,+} \right) &\leftrightarrow W^{(0)}_{+} + W^{(1)}_{++} = +t_0 + \frac{t_1}{2} \\
	\ket{\psi^0_{+-}} = \frac{1}{\sqrt{2}} \left( \ket{0,+} - \ket{1,+} \right) &\leftrightarrow W^{(0)}_+ + W^{(1)}_{+-} = +t_0 - \frac{t_1}{2} \\
	\ket{\psi^0_{-+}} = \frac{1}{\sqrt{2}} \left( \ket{0,-} - \ket{1,-} \right) &\leftrightarrow W^{(0)}_- + W^{(1)}_{-+} = -t_0 + \frac{t_1}{2} \\
	\ket{\psi^0_{--}} = \frac{1}{\sqrt{2}} \left( \ket{0,-} + \ket{1,-} \right) &\leftrightarrow W^{(0)}_- + W^{(1)}_{--} = -t_0 - \frac{t_1}{2}
\end{align*}

\textbf{Perturbation à l'ordre 2}\\
On calcule maintenant le déplacement des vecteurs propres à l'ordre 2, $\ket{\psi^1_{\pm \pm}}$.
Ici, la formule \eqref{ordre2} donne
\begin{equation}
	\ket{\psi^1_{\varepsilon_1 \varepsilon_2}} = \sum_{(\varepsilon_1',\varepsilon_2') \neq (\varepsilon_1, \varepsilon_2)} \frac{\braket{\psi^0_{\varepsilon_1' \varepsilon_2'}| H_1 | \psi^0_{\varepsilon_1 \varepsilon_2}}}{E_{\varepsilon_1 \varepsilon_2} - E_{\varepsilon_1'\varepsilon_2'}} \ket{\psi^0_{\varepsilon_1' \varepsilon_2'}}
\end{equation}
On obtient
\begin{align*}
	\ket{\psi^1_{++}} = - \frac{t_1}{4t_0} \ket{\psi^0_{-+}} \\
	\ket{\psi^1_{+-}} = + \frac{t_1}{4t_0} \ket{\psi^0_{--}} \\
	\ket{\psi^1_{-+}} = + \frac{t_1}{4t_0} \ket{\psi^0_{++}} \\
	\ket{\psi^1_{--}} = - \frac{t_1}{4t_0} \ket{\psi^0_{+-}}
\end{align*}
Soit, dans la base $\ket{0,+},\ket{0,-},\ket{1,+},\ket{1,-}$ :
\begin{align}
	\ket{++} = \frac{1}{\sqrt{2}} \left\{1,-\frac{\text{t1}}{4},1,\frac{\text{t1}}{4}\right\} \\
	\ket{+-} = \frac{1}{\sqrt{2}} \left\{1,\frac{\text{t1}}{4},-1,\frac{\text{t1}}{4}\right\} \\
	\ket{-+} = \frac{1}{\sqrt{2}} \left\{\frac{\text{t1}}{4},1,\frac{\text{t1}}{4},-1\right\} \\
	\ket{--} = \frac{1}{\sqrt{2}} \left\{-\frac{\text{t1}}{4},1,\frac{\text{t1}}{4},1\right\}
\end{align}



\begin{align}
	\ket{\psi^0_{+++}} = \frac{1}{\sqrt{2}}\left( \ket{0,++} + \ket{1,++} \right) \\
	\ket{\psi^0_{++-}} = \frac{1}{\sqrt{2}}\left( \ket{0,++} - \ket{1,++} \right) \\
	\ket{\psi^0_{+-+}} = \frac{1}{\sqrt{2}}\left( \ket{0,+-} - \ket{1,+-} \right) \\
	\ket{\psi^0_{+--}} = \frac{1}{\sqrt{2}}\left( \ket{0,+-} + \ket{1,+-} \right) \\
	\ket{\psi^0_{-++}} = \frac{1}{\sqrt{2}}\left( \ket{0,-+} + \ket{1,-+} \right) \\
	\ket{\psi^0_{-+-}} = \frac{1}{\sqrt{2}}\left( \ket{0,-+} - \ket{1,-+} \right) \\
	\ket{\psi^0_{--+}} = \frac{1}{\sqrt{2}}\left( \ket{0,--} - \ket{1,--} \right) \\
	\ket{\psi^0_{---}} = \frac{1}{\sqrt{2}}\left( \ket{0,--} + \ket{1,--} \right)
\end{align}

\begin{align}
	\ket{\psi^1_{+++}} = -\frac{t_2}{4 t_1} \ket{\psi^0_{+-+}} + \frac{t_2}{8 t_0}\left( +\ket{\psi^0_{-++}} - \ket{\psi^0_{--+}} \right) \\
	\ket{\psi^1_{++-}} = +\frac{t_2}{4 t_1} \ket{\psi^0_{+--}} + \frac{t_2}{8 t_0} \left( -\ket{\psi^0_{-+-}} + \ket{\psi^0_{---}} \right) \\
	\ket{\psi^1_{+-+}} = +\frac{t_2}{4 t_1} \ket{\psi^0_{+++}} +  \frac{t_2}{8 t_0} \left( -\ket{\psi^0_{-++}} + \ket{\psi^0_{--+}} \right) \\
	\ket{\psi^1_{+--}} = -\frac{t_2}{4 t_1} \ket{\psi^0_{++-}} +  \frac{t_2}{8 t_0} \left( +\ket{\psi^0_{-+-}} - \ket{\psi^0_{---}} \right) \\
	\ket{\psi^1_{-++}} = -\frac{t_2}{4 t_1} \ket{\psi^0_{--+}} + \frac{t_2}{8 t_0}\left( -\ket{\psi^0_{+++}} + \ket{\psi^0_{+-+}} \right) \\
	\ket{\psi^1_{-+-}} = +\frac{t_2}{4 t_1} \ket{\psi^0_{---}} + \frac{t_2}{8 t_0}\left( +\ket{\psi^0_{++-}} - \ket{\psi^0_{+--}} \right) \\
	\ket{\psi^1_{--+}} = +\frac{t_2}{4 t_1} \ket{\psi^0_{-++}} + \frac{t_2}{8 t_0}\left( +\ket{\psi^0_{+++}} - \ket{\psi^0_{+-+}} \right) \\
	\ket{\psi^1_{---}} = -\frac{t_2}{4 t_1} \ket{\psi^0_{-+-}} + \frac{t_2}{8 t_0}\left( -\ket{\psi^0_{++-}} + \ket{\psi^0_{+--}} \right)
\end{align}

\begin{equation}
	\ket{\psi^1_{\varepsilon_1 \varepsilon_2 \varepsilon_3}} = (-1)^{\bar{\varepsilon}_2 + \varepsilon_3} \frac{t_2}{4 t_1} \ket{\psi^0_{\varepsilon_1 \bar{\varepsilon}_2 \varepsilon_3}} + (-1)^{\varepsilon_1 + \varepsilon_3} \frac{t_2}{8 t_0} \left( \ket{\psi^0_{\bar{\varepsilon}_1 \varepsilon_2 \varepsilon_3}} - \ket{\psi^0_{\bar{\varepsilon}_1 \bar{\varepsilon}_2 \varepsilon_3}} \right)
\end{equation}
\end{document}